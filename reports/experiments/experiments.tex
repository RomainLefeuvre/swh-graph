\documentclass[11pt,a4paper]{article}

\usepackage[english]{babel}
\usepackage[a4paper,left=2cm,right=2cm,top=2.5cm,bottom=2.5cm]{geometry}
\usepackage{booktabs}
\usepackage{hyperref}
\usepackage{minted}
\usepackage{parskip}
\usepackage{siunitx}

\title{Google Summer of Code 2019}
\author{Thibault Allançon}
\date{April 8, 2019}

\begin{document}

\maketitle

Early experiments running WebGraph framework on the Software Heritage datasets.

\section{Environment}

Docker environment and compression script can be found here:
\url{https://forge.softwareheritage.org/source/swh-graph/browse/master/dockerfiles/}.

\section{Datasets analysis}

\begin{center}
    \begin{tabular}{@{} l *4r @{}}
        \toprule
        \multicolumn{1}{c}{} &
            \textbf{\mintinline{text}{.nodes.csv.gz}} &
            \textbf{\mintinline{text}{.edges.csv.gz}} &
            \textbf{\# of nodes} & \textbf{\# of edges} \\
        \midrule
        \texttt{release\_to\_obj}
            & 344M & 382M & \num{16222788} & \num{9907464} \\
        \texttt{origin\_to\_snapshot}
            & 1.3G & 3.7G & \num{112564374} & \num{194970670} \\
        \texttt{dir\_to\_rev}
            & 745M & 12G & \num{35399184} & \num{481829426} \\
        \texttt{snapshot\_to\_obj}
            & 3.5G & 21G & \num{170999796} & \num{831089515} \\
        \texttt{rev\_to\_rev}
            & 22G & 33G & \num{1117498391} & \num{1165813689} \\
        \texttt{rev\_to\_dir}
            & 41G & 48G & \num{2047888941} & \num{1125083793} \\
        \texttt{dir\_to\_dir}
            & 95G & 1.3T & \num{4805057515} & \num{48341950415} \\
        \texttt{dir\_to\_file}
            & 180G & 3T & \num{9231457233} & \num{112363058067} \\
        \midrule
        Entire graph (\texttt{all})
            & 340G & 4.5T & \num{11595403407} & \num{164513703039} \\
        \bottomrule
    \end{tabular}
\end{center}

\section{Individual datasets compression}

The first experiments were done on individual datasets.

\subsection{Results}

Datasets were compressed on different VM (depending on availability):

\begin{itemize}
    \item \textit{(sexus)} 1TB of RAM and 40vCPU: \mintinline{text}{dir_to_dir}
    \item \textit{(monster)} 700GB of RAM and 72vCPU:
        \mintinline{text}{dir_to_file}
    \item \textit{(chub)} 2TB of RAM and 128vCPU: all the others datasets
\end{itemize}

Note: the results may vary because random permutations are used in the graph
compression process.

\begin{center}
    \begin{tabular}{@{} l *4r @{}}
        \toprule
        \multicolumn{1}{c}{} &
            \textbf{compr ratio} & \textbf{bit/edge} & \textbf{compr
            size\footnotemark} \\
        \midrule
        \texttt{release\_to\_obj} & 0.367 & 9.573 & 23M \\
        \texttt{origin\_to\_snapshot} & 0.291 & 8.384 & 140M \\
        \texttt{dir\_to\_rev} & 0.07 & 1.595 & 120M & \\
        \texttt{snapshot\_to\_obj} & 0.067 & 1.798 & 253M \\
        \texttt{rev\_to\_rev} & 0.288 & 9.063 & 2.2G \\
        \texttt{rev\_to\_dir} & 0.291 & 9.668 & 2.6G \\
        \texttt{dir\_to\_dir} & 0.336 & 10.178 & 61G \\
        \texttt{dir\_to\_file} & 0.228 & 7.054 & 97G \\
        \midrule
        Entire graph (estimated) & & & 163G \\
        \bottomrule
    \end{tabular}
\end{center}

\footnotetext{calculated as: size of \mintinline{bash}{*.graph} + size of
\mintinline{bash}{*.offsets}}

\subsection{Timings}

\begin{center}
    \begin{tabular}{@{} l *6r @{}}
        \toprule
        \multicolumn{1}{c}{} &
            \textbf{MPH} &
            \textbf{BVGraph} &
            \textbf{Symmetrized} &
            \textbf{LLP} &
            \textbf{Permutation} &
            \textbf{Total} \\
        \midrule
        \texttt{release\_to\_obj}
            & 14s & 25s & 18s & 8min & 10s & \textbf{9min} \\
        \texttt{origin\_to\_snapshot}
            & 1min & 5min & 3min & 1h30 & 1min & \textbf{1h40} \\
        \texttt{dir\_to\_rev}
            & 56s & 22min & 6min & 41min & 2min & \textbf{1h13} \\
        \texttt{snapshot\_to\_obj}
            & 3min & 22min & 8min & 2h50 & 5min & \textbf{3h30} \\
        \texttt{rev\_to\_rev}
            & 11min & 56min & 24min & 31h52 & 20min & \textbf{33h42} \\
        \texttt{rev\_to\_dir}
            & 20min & 1h & 30min & 52h45 & 23min & \textbf{55h} \\
        \bottomrule
    \end{tabular}
\end{center}

\vspace{0.5cm}

For the \mintinline{text}{dir_to_*} datasets we decided not use LLP algorithm
because it would take too long, and instead used a BFS traversal order for the
node re-ordering. This allows \textbf{much} faster computation and yields
similar results (thanks to our graph topology).

\vspace{0.5cm}

\begin{center}
    \begin{tabular}{@{} l *5r @{}}
        \toprule
        \multicolumn{1}{c}{} &
            \textbf{MPH} &
            \textbf{BVGraph} &
            \textbf{BFS} &
            \textbf{Permutation} &
            \textbf{Total} \\
        \midrule
        \texttt{dir\_to\_dir}
            & 4h36 & 50h & 4h44 & 12h38 & \textbf{72h} \\
        \texttt{dir\_to\_file}
            & 3h07 & 101h & 17h18 & 20h38 & \textbf{142h} \\
        \bottomrule
    \end{tabular}
\end{center}

\subsection{Memory usage}

Memory usage monitoring during the compression process:

\begin{center}
    \begin{tabular}{@{} l c @{}}
        \toprule
        \multicolumn{1}{c}{} &
            \textbf{Maximum resident set size} \\
        \midrule
        \texttt{release\_to\_obj} & 11G \\
        \texttt{origin\_to\_snapshot} & 15G \\
        \texttt{dir\_to\_rev} & 22G \\
        \texttt{snapshot\_to\_obj} & 23G \\
        \texttt{rev\_to\_rev} & 86G \\
        \texttt{rev\_to\_dir} & 154G \\
        \texttt{dir\_to\_dir} & 345G \\
        \texttt{dir\_to\_file} & 764G \\
        \midrule
        Entire graph (estimated) & 1.4T \\
        \bottomrule
    \end{tabular}
\end{center}

\section{Entire graph compression}

After studying feasibility on the individual datasets and estimating the final
results, we assembled the entire graph into a single dataset and launched the
compression process on it.

\subsection{Results}

Two different VM where used depending on the compression step:

\begin{itemize}
    \item \textit{(monster)} 700GB of RAM and 72vCPU: for the BVGraph step.
    \item \textit{(rioc)} 3TB of RAM and 48vCPU: all the other steps.
\end{itemize}

The reason to use monster instead of rioc for the BVGraph step was because the
I/O on rioc was too slow for the job to complete within the time limit allowed
on the cluster.

\begin{center}
    \begin{tabular}{@{} l *3r @{}}
        \toprule
        \multicolumn{1}{c}{} &
            \textbf{compr ratio} & \textbf{bit/edge} & \textbf{compr size} \\
        \midrule
        \texttt{all} & 0.158 & 4.913 & 101G \\
        \bottomrule
    \end{tabular}
\end{center}

\subsection{Timings}

\begin{center}
    \begin{tabular}{@{} l *5r @{}}
        \toprule
        \multicolumn{1}{c}{} &
            \textbf{MPH} &
            \textbf{BVGraph} &
            \textbf{BFS} &
            \textbf{Permutation} &
            \textbf{Total} \\
        \midrule
        \texttt{all}
            & 3h30 & 103h & 9h52 & 24h39 & \textbf{141h} \\
        \bottomrule
    \end{tabular}
\end{center}

Extra steps (more specific to our needs):

\begin{center}
    \begin{tabular}{@{} l *2r @{}}
        \toprule
        \multicolumn{1}{c}{} &
            \textbf{Stats} &
            \textbf{Transpose} \\
        \midrule
        \texttt{all}
            & 3h58 & 21h33 \\
        \bottomrule
    \end{tabular}
\end{center}

\subsection{Memory usage}

\begin{center}
    \begin{tabular}{@{} l c @{}}
        \toprule
        \multicolumn{1}{c}{} &
            \textbf{Maximum resident set size} \\
        \midrule
        \texttt{all} & 1057G \\
        \bottomrule
    \end{tabular}
\end{center}

\end{document}
